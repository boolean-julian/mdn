% Please compile using lualatex
% Looks better with Mozilla's font "Fira" installed (https://github.com/mozilla/Fira/)
\documentclass[xcolor = dvipsnames, notheorems]{beamer}
\usepackage{amssymb}
\usepackage{amsmath}
\usepackage{mathtools}
\usepackage{enumitem}
\usepackage{graphicx}
\usepackage{mathrsfs}
\usepackage[ngerman]{babel}

\title{Methoden der Numerik}
\date{\today}
\author{Christina Eilers, Julian Lüken}
\institute{Mathematisches Institut Göttingen}

\usetheme[numbering=none]{metropolis}
\renewcommand{\emph}[1]{\textcolor[HTML]{EB801A}{#1}}
\newcommand{\qedcolor}[1]{\textcolor[HTML]{22373A}{#1}}
\setbeamercolor{section in head/foot}{fg=normal text.bg, bg=structure.fg}
\setbeamercovered{transparent}

\setenumerate{align=left, leftmargin=*}
\setitemize{align=left, leftmargin=*, label=\emph{$\bullet$}}

\DeclareMathOperator{\rel}{\sim_R}

\newcommand{\vth}{\vspace{4pt}}
\theoremstyle{definition}
\newtheorem{definition}	{Definition:\vth}
\newtheorem{example}	{Beispiel:\vth}
\newtheorem{theorem}	{Satz:\vth}
\newtheorem{corollary}	{Korollar:\vth}
\newtheorem{remark}		{Bemerkung:\vth}

\renewenvironment{proof}[1][Beweis. ]{\setbeamercovered{invisible}\textbf{#1}}{\begin{flushright}\qedcolor{$\blacksquare$}\end{flushright}\setbeamercovered{transparent}}

\newenvironment{proofbegin}[1][Beweis. ]{\setbeamercovered{invisible}\textbf{#1}}

\newenvironment{proofend}{}{\begin{flushright}\qedcolor{$\blacksquare$}\end{flushright}\setbeamercovered{transparent}}


\begin{document}
% Title page
\begin{frame}
	\maketitle
\end{frame}

\section{Aufgabe 1 - Wärmegleichung}
\begin{frame}
\frametitle{Wärmegleichung}
	Die Wärmegleichung lautet
	$$ \frac{\partial u}{\partial t} - \alpha \nabla^2 u$$
	Mit $u: \Omega \times \mathbb{R}^+ \rightarrow \mathbb{R}$ mit folgenden Randbedingungen:
	\begin{itemize}
		\item $ u(x,t) = R$ für $ x \in \partial \Omega$	
		\item $ u(x,0) = f(x) $, wobei $f$ beliebig aber fest.
	\end{itemize}
\end{frame}

\begin{frame}
\frametitle{Diskretisierung der Wärmegleichung}
	Nimm endlich viele, äquidistante Stellen aus $\Omega$, sodass Folgen entstehen mit $x_i = ih + x_0$ und $y_j = jh + y_0$. Wähle zusätzlich für die Zeit $t_k = k\Delta t + t_0$ Wir schreiben $U_{i,j}^k$ für die $i,j$-te Stelle zum Zeitpunkt $k$. Für jedes $k \in \mathbb{N}$ entsteht eine $m \times m$ Matrix. Zum Zeitpuntk $k$ haben wir dann:
	$$
	\begin{pmatrix}
		u^k_{0,0}& 		u^k_{0,1}& \cdots & u^k_{0,m} 	\\
		u^k_{1,0}&		\ddots&			&\vdots	\\
		\vdots & & &	\\
		u^k_{m,0}&	\cdots	&			 & u^k_{m,m}
	\end{pmatrix}
	$$
\end{frame}
\begin{frame}
\frametitle{Diskretisierung der Wärmegleichung}
	Schreibe diese Matrix als Vektor, damit wir einen linearen Operator in Form einer Matrix darauf anwenden können, folgendermaßen:
	$$ u^k =
	\begin{pmatrix}
		u^k_{0,0} \\
		u^k_{0,1} \\
		\vdots \\
		u^k_{0,m} \\
		u^k_{1,0} \\
		u^k_{1,1} \\
		\vdots \\
		u^k_{m,m}
	\end{pmatrix}
	$$
\end{frame}
\begin{frame}
\frametitle{Diskretisierung der Wärmegleichung}
	Aus der Taylor-Entwicklung folgt für die Ableitung nach $t$
	$$\frac{\partial u}{\partial t}(x,y,t) = \frac{u(x,y,t+\Delta t) - u(x,y,t)}{\Delta t} + O(\Delta t^2)$$
	und für den Laplace-Operator
	\begin{multline*}
		\nabla^2u(x,y,t) = \frac{1}{4h^2} \bigg(u(x-h,y,t) + u(x+h,y,t) \\+ u(x,y-h,t) + u(x,y+h,t) - 4u(x,y,t) \bigg) + O(h^4)
	\end{multline*}
\end{frame}
\begin{frame}
\frametitle{Diskretisierung der Wärmegleichung}
	Für den diskreten Fall (und unter der Annahme, dass bei den Gleichungen von vorhin der Fehlerterm für hinreichend kleine $h$ und $\Delta t$ wegfällt), erhalten wir folgende Matrix:
	$$
		A = \frac{\alpha}{4h^2}\bigg(\text{tridiag}(1,-4,1) + G \bigg) \in \mathbb{R}^{m^2 \times m^2}
	$$
	wobei
	$$
	G =\begin{pmatrix}
			\mathbf{0} 	& \mathbf{I} 	& \mathbf{0} 	& 				& \cdots 		&				& \mathbf{0} 	\\
			\mathbf{I}	& \mathbf{0} 	& \mathbf{I} 	& 				&				& 				&			 	\\
			\mathbf{0}	& \mathbf{I}	&				& 				& 				&				&				\\
			\vdots		&				&				& 				& \ddots		&				& \vdots		\\
						&				&				&				&				& \mathbf{0}	& \mathbf{I}	\\
			\mathbf{0}	&				&				&				&				& \mathbf{I}	& \mathbf{0}	\\
		\end{pmatrix}
	$$
	eine Blockmatrix mit $m \times m$ Einheitsmatrizen auf den Nebendiagonalen.
\end{frame}


\begin{frame}
\frametitle{Diskretisierung der Wärmegleichung}
	Durch Umstellen der Wärmegleichung mit $A$ von vorhin können wir durch folgende Iterationsvorschrift mit gegebenen Startwerten $u^0$ den Verlauf der Wärmegleichung simulieren:
	$$u^{k+1} = (A+I)u^k$$
	Mehrfachanwendung des Operators steht für größeres $\Delta t$. Wählt man mehr Samples, so wird $h$ kleiner.
\end{frame}



\section{Aufgabe 2 - Newton-Verfahren}
\begin{frame}
\frametitle{Lösungen}
	\footnotesize
	\begin{center}
		\begin{tabular}{r || l | l | l}
			& klassisch & BFGS & Broyden\\ \hline \hline
			$f(x)$ &$2291.020503182124$ &$2291.0205031821224$ &$2291.020503182123$\\
			$\nabla f(x)$	&$\begin{pmatrix}
													0 \\
													-1.137 \\
													0.941 \\
													-0.284 \\
													5.684 \\
													0 \\
													0
							\end{pmatrix} \cdot 10^{-13}$
							&$\begin{pmatrix}
													-2.274 \\ 
													67.075 \\
													21.245 \\
													-7.248 \\
													37.232 \\
													-2.256 \\
													-2.487
							\end{pmatrix} \cdot 10^{-13}$
							&$\begin{pmatrix}
													-3.411 \\
													-14.780 \\
													-4.814 \\
													1.705 \\
													86.118 \\
													-3.606 \\
													0.249 \\
							\end{pmatrix} \cdot 10^{-13}$\\
		\end{tabular}
	\end{center}
	\normalsize
\end{frame}
\begin{frame}
\frametitle{Hesse-Matrizen}
	\tiny
		$$H^{-1} = \begin{pmatrix}
			&2.151 &0.598 &3.538 &16.956 &-0.078 &-2.615 &-0.039 \\
			&0.598 &3.288 &0.121 &22.659 &0.310 &13.016 &-0.011 \\
			&3.538 &0.121 &13.622 &45.657 &-0.220 &-8.099 &-0.260 \\
			&16.956 &22.659 &45.657 &323.998 &1.294 &58.392 &-0.897 \\
			&-0.078 &0.310 &-0.220 &1.294 &0.148 &2.717 &0.001 \\
			&-2.615 &13.016 &-8.099 &58.392 &2.717 &93.675 &0.047 \\
			&-0.039 &-0.011 &-0.260 &-0.897 &0.001 &0.047 &3.256 \\
		\end{pmatrix} \cdot 10^{-3}$$
		$$H^{-1}_\text{BFGS} = \begin{pmatrix}
			&1.992 &0.337 &3.517 &14.757 &-0.068 &-3.207 &-0.025 \\
			&0.337 &2.532 &0.301 &17.282 &0.273 &10.557 &-0.003 \\
			&3.517 &0.301 &13.449 &46.290 &-0.189 &-7.443 &-0.278 \\
			&14.757 &17.282 &46.290 &284.436 &1.163 &42.147 &-0.838 \\
			&-0.068 &0.273 &-0.189 &1.163 &0.135 &2.584 &-0.014 \\
			&-3.207 &10.557 &-7.443 &42.147 &2.584 &80.445 &0.562 \\
			&-0.025 &-0.003 &-0.278 &-0.838 &-0.014 &0.562 &3.115
		\end{pmatrix} \cdot 10^{-3}$$
		$$H^{-1}_\text{Broyden} = \begin{pmatrix}
			&1.003 &0.353 &2.534 &10.787 &-0.024 &-0.808 &-0.305 \\
			&0.016 &0.559 &1.185 &4.623 &0.082 &4.472 &-0.130 \\
			&0.035 &0.129 &9.330 &30.184 &-0.035 &-1.636 &-0.773 \\
			&0.301 &4.001 &62.421 &234.475 &-0.381 &2.729 &-5.924 \\
			&-0.171 &0.436 &-1.168 &-0.179 &0.145 &2.657 &0.025 \\
			&1.258 &-0.488 &22.002 &72.069 &0.185 &7.744 &-1.962 \\
			&0.009 &-0.057 &-1.015 &-3.702 &0.007 &-0.004 &1.718
		\end{pmatrix} \cdot 10^{-3}$$
	\normalsize
\end{frame}
\end{document}