% Please compile using lualatex
% Looks better with Mozilla's font "Fira" installed (https://github.com/mozilla/Fira/)
\documentclass[xcolor = dvipsnames, notheorems]{beamer}
\usepackage{amssymb}
\usepackage{amsmath}
\usepackage{mathtools}
\usepackage{enumitem}
\usepackage{graphicx}
\usepackage{mathrsfs}
\usepackage[ngerman]{babel}

\title{Methoden der Numerik}
\date{\today}
\author{Christina X., Julian Lüken}
\institute{Mathematisches Institut Göttingen}

\usetheme[numbering=none]{metropolis}
\renewcommand{\emph}[1]{\textcolor[HTML]{EB801A}{#1}}
\newcommand{\qedcolor}[1]{\textcolor[HTML]{22373A}{#1}}
\setbeamercolor{section in head/foot}{fg=normal text.bg, bg=structure.fg}
\setbeamercovered{transparent}

\setenumerate{align=left, leftmargin=*}
\setitemize{align=left, leftmargin=*, label=\emph{$\bullet$}}

\DeclareMathOperator{\rel}{\sim_R}

\newcommand{\vth}{\vspace{4pt}}
\theoremstyle{definition}
\newtheorem{definition}	{Definition:\vth}
\newtheorem{example}	{Beispiel:\vth}
\newtheorem{theorem}	{Satz:\vth}
\newtheorem{corollary}	{Korollar:\vth}
\newtheorem{remark}		{Bemerkung:\vth}

\renewenvironment{proof}[1][Beweis. ]{\setbeamercovered{invisible}\textbf{#1}}{\begin{flushright}\qedcolor{$\blacksquare$}\end{flushright}\setbeamercovered{transparent}}

\newenvironment{proofbegin}[1][Beweis. ]{\setbeamercovered{invisible}\textbf{#1}}

\newenvironment{proofend}{}{\begin{flushright}\qedcolor{$\blacksquare$}\end{flushright}\setbeamercovered{transparent}}


\begin{document}
% Title page
\begin{frame}
	\maketitle
\end{frame}

\section{Aufgabe 1 - Wärmegleichung}
\begin{frame}
\frametitle{Wärmegleichung}
	Die Wärmegleichung lautet
	$$ \frac{\partial u}{\partial t} -\nabla^2 u$$
	Mit $u: \Omega \times \mathbb{R}^+ \rightarrow \mathbb{R}$ mit folgenden Randbedingungen:
	\begin{itemize}
		\item $ u(x,t) = R$ für $ x \in \partial \Omega$	
		\item $ u(x,0) = f(x) $, wobei $f$ beliebig aber fest.
	\end{itemize}
\end{frame}

\begin{frame}
\frametitle{Diskretisierung, Variante 1}
	Nimm endlich viele, äquidistante Stellen aus $\Omega$, sodass Folgen entstehen mit $x_i = ih + x_0$ und $y_j = jh + y_0$. Wähle zusätzlich für die Zeit $t_k = k\Delta t + t_0$ Wir schreiben $U_{i,j}^k$ für die $i,j$-te Stelle zum Zeitpunkt $k$. O.B.d.A. nehmen wir an, dass wir einen quadratischen Bereich diskretisieren, d.h. für jedes $k \in \mathbb{N}$ entsteht eine $m \times m$ Matrix. Zum Zeitpuntk $k=0$ haben wir dann:
	$$
	\begin{pmatrix}
		u^0_{0,0}& 		u^0_{0,1}& \cdots 	\\
		u^0_{1,0}&		\cdots&				\\
		\cdots &		&			u^0_{m,m}
	\end{pmatrix}
	$$
\end{frame}

\end{document}